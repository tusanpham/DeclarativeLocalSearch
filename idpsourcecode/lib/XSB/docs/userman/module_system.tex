% attempt to re-document the XSB module system

\section{The Module System of XSB} \label{Modules}

XSB has been designed as a module-oriented Prolog system.  Modules
provide a step towards {\em logic programming ``in the large''} that
facilitates the construction of large programs or projects from
components that are developed, compiled and tested separately.  Also,
module systems support the principle of information hiding and can
provide a basis for data abstraction.  And modules form the basis for
XSB's implementation of its standard predicates.

The module system of XSB is {\em file based} -- one module per file --
and {\em flat} -- modules cannot be nested.  In addition, XSB's module
system is essentially {\em atom-based} (or {\em
  structure-symbol-based}), where any symbol in a module can be
imported, exported or be a local symbol, as opposed to a {\em
  predicate-based} one where this can be done only for predicate
symbols~\footnote{Operator symbols can be exported as any other
  symbols, but their precedence must be redeclared in the importing
  module.}.  Every structure symbol (and thus structured term) is
associated with a module, and structure symbols with the same name but
in different modules are different symbols and thus do not unify.  As
we will discuss, this leads to certain differences of XSB's module
system from those of some other Prologs, and to certain
incompatibilities with the (proposed) ISO standard for modules (which
is not supported by most Prologs).  At the same time, XSB's module
system has enough commonalities with those of other Prologs to be able
to support the Prolog Commons libraries.

In XSB (as in all Prolog systems) predicate definitions (aka Clauses)
are associated with structure symbols.  A predicate is a structure
symbol with an associated definition.  Predicates are either static or
dynamic.  Static predicates get their definitions from source code
files that are compiled and loaded into memory.  Dynamic predicates
get their definitions from the execution of the builtin meta-predicate
{\tt assert/1} (and friends).

In XSB every structure symbol is associated with a module.  A term is
said to be in the module of its main structure symbol.  Terms in
different modules are different terms and do {\bf not} unify.  So two
terms whose main structure symbols (or any structure symbols) have the
same name but different modules, are different terms and do not unify.
So, for example, terms printed as {\tt p(a,b)} and {\tt p(a,b)} would
not unify if the first structure symbol named {\tt p/2} is in a
different module from the second structure symbol named {\tt p/2}.

The ``default'' module is named {\tt usermod}.  Whenever a term is
constructed, and a module is not explicitly provided, {\tt usermod} is
the module used.  For example, when {\tt functor/3} (or {\tt univ/2})
is used to construct a term, that term is put in {\tt usermod}.  Any
term that is read from a file (or at the top-level prompt) is put in
{\tt usermod}.  All (usual) XSB source files, when compiled and
loaded, define predicates in {\tt usermod}.

So how are terms that are not in usermod constructed?  The most
important use of modules by far is to organize predicates (and thus
their definitions.)  So a module is associated with a set of predicate
definitions, which in XSB is a Prolog source file, a file with {\tt
  .P} extension.  In XSB, a source file is compiled (to a {\tt .xwam}
file) and then loaded into memory to provide definitions for the
predicates with clauses in that file.  For ``usual'' XSB source files,
all the defined predicates are in {\tt usermod}.  However when a
source file includes an export directive, such as:

\demo{:- export $Pred/Arity$.}

\noindent the definitions in that source file will be interpreted as
defining predicates in a module.  The name of the module is the name
of the XSB source file (without the extension {\tt .P}).  Predicates
that have definitions in such a file will all be put in the module of
that name.  A predicate that is exported must be defined in the
source file and will be made available for use in other source files,
when imported.  An import directive, such as:

\demo{:- import $Pred/Arity$ from $Module$.}

\noindent in another source file allows its definitions to use that
exported predicate.  For example, the file:
\begin{verbatim}
%% file: mod1.P
:- export p/2.

p(a,b).
p(X,Y) :- q(X,Y).

q(b,c).
\end{verbatim}
when compiled and loaded, defines a predicate {\tt p/2} in {\tt mod1}
(i.e., {\tt p/2} terms that define the facts have their main structure
symbols put in module {\tt mod1}, and the code implementing those
clauses are associated with that structure symbol in that module.)  It
also defines a predicate {\tt q/2} in the same module.  (And, of
course, the call to {\tt q(X,Y)} in the body of the rule for {\tt p/2}
is also put in that same module.)  The predicate {\tt p/2} is exported
and is thus available for use by other code.

For example, we can create another file (not a module in this case),
which uses the definition of {\tt p/2} above:
\begin{verbatim}
%% file: my_code.P
:- import p/2 from mod1.

q(X,Y) :- p(X,Y).
\end{verbatim}
\noindent Here there is no {\tt export} directive, so all definitions
in this file will go into module {\tt usermod}.  The clause here
defines {\tt q/2} in {\tt usermod}, which is a different predicate
from the {\tt q/2} defined above in the module {\tt mod1}.  The {\tt
  import} of {\tt p/2} in this file causes the {\tt p(X,Y)} term in
the body of the rule for {\tt q/2} to be interpreted as being in
module {\tt mod1}.  Thus, when this file is compiled and loaded, {\tt
  q/2} is defined in {\tt usermod} and its code calls {\tt p/2} in
module {\tt mod1}.

A module source file may want to access a predicate defined in {\tt
  usermod}, which can be done by explicitly importing the predicate
from {\tt usermod}.

There are situations in which a programmer wants to explicitly provide
a module name to ``override'' the module associated with a term.  For
example, one might want to call the goal {\tt p(X,Y)} to invoke the
code associated with {\tt p/2} in module {\tt mod1} at the top level,
regardless of what module the {\tt p/2} structure symol is associated
with.  In this case, one can write:

\demo{| ?- call(mod1:p(X,Y)).}

\noindent Here {\tt call} will construct the term {\tt p(X,Y)} with
structure {\tt p/2} in module {\tt mod1} (ignoring the module associated
with the {\tt p/2} structure symbol) and then call that term, which
accesses the code of {\tt p/2} in module {\tt mod1}.  In this
particular case the original term {\tt mod1:p(X,Y)} had the {\tt p/2}
structure in {\tt usermod}, since that's where the top-level read puts
it.  But {\tt call/1} interprets this term (with main structure symbol
{\tt :/2}) as a coercion of the term {\tt p(X,Y)} into the module {\tt
  mod1}.  In XSB, in most contexts in which a term is interpreted as a
goal, the syntax of {\tt Mod:Goal} is interpreted as a coercion of
term {\tt Goal} into the module {\tt Mod}.  And in fact, the top-level
goal:

\demo{| ?- mod1:p(X,Y).}

\noindent is equivalent to the goal above.

And instead of:
\begin{verbatim}
:- import pr/2 from mod3.
...
q(X,Y) :- ... pr(X,Z), ....
\end{verbatim}
one can directly write:
\begin{verbatim}
q(X,Y) :- ... mod3:pr(X,Z), ....
\end{verbatim}

In general, the use of {\tt import} is recommended, even though it may
sometimes be more verbose.  The use of {\tt imort} allows for better
visibility and easier analysis of module dependencies.

In XSB, the declaration:

\demo{:- module($filename$,[$sym_1$, \ldots, $sym_l$.]).}

\noindent
is syntactic sugar for:

\demo{:- export $sym_1$, \ldots, $sym_l$. }

\noindent
as long as the $filename$ is the same as the name of the file in which
it was contained.  Similarly,

\demo{:- use\_module($module$,[$sym_1$, \ldots, $sym_l$.]).}

\noindent
is treated as semantically equivalent to 

\demo{:- import $sym_1$, \ldots, $sym_n$ from $module$. }

\noindent
Accordingly, {\tt use\_module/2} and {\tt module/1} can be used
interchangibly with {\tt import/2} and {\tt export/1}.  However the
declaration

\demo{:- use\_module($module$).}

\noindent
which is often used in other Prolog systems, is {\em not} equivalent
to an XSB import statement, as each XSB import statement must
explicitly declare a list of predicates that are used from each
module.  Such a declaration will raise a compilation error.

The declaration 

\demo{:- import $sym$ from $module$ as $sym'$. }

\noindent
allows a predicate to be imported from a module, but renamed as $sym'$
within the importing module.  In this case the structure symbol $sym'$
is placed in the current module and its code pointer is identified
with that of the structure symbol $sym$ in module $module$.  Such a
feature is useful when porting a library written for another Prolog
(e.g. a constraint library) to XSB.  It is also useful when one wants
to import two predicates with the same name from different modules.
In that case (at least) one of the names needs to be changed on
import.

For modules, the base file name is stored in its byte code ({\tt
  .swam} file, so that renaming a byte-code file for a mule may cause
problems, as the renaming will not affect the information within the
byte-code file.  However, byte code files generated for non-modules
can be safely renamed.

\subsection{How the Compiler Determines the Module of a Term}

When XSB compiles a source code file, it must determine the module for
every term it encounters.  For non-module source files (i.e., those
with an {\tt export} directive), all terms are associated with {\tt
  usermod} except for those whose structure symbols are imported.  Any
occurrence of an imported structure symbol is associated with the
module from which it is imported.

For module source code files, i.e., those containing at least one {\tt
  export} directive, the process of determining the module of a
structure symbol is more complicated.  The idea is that all terms in
the source file that refer to predicates are placed in the module of
the file, and all terms that are record structures are by default
placed in {\tt usermod}.  All occurrences of the same structure symbol
in a file are normally associated with the same module\footnote{but
  see {\tt import .. as ..} for an exception.}  So if a structure
symbol appears both as a predicate symbol (e.g., as a subgoal in the
body of a rule) and as a record structure (perhaps to be passed to
some other predicate to later be called), both occurrences will be
associated with the current module.  Of course, imported structure
symbols are associated with the module from which they are imported.

The compiler recognizes as predicate symbols any symbol that:
\begin{enumerate}
\item appears as the main structure symbol in the head of a rule,

\item appears as a subgoal in the body of a rule,

\item appears as the main structure symbol of terms passed to known
  meta-predicates, such as {\tt assert/1} and {\tt retract},

\item is declared as {\tt dynamic}.
\end{enumerate}

Otherwise a structure symbol is associated with {\tt usermod}.

Note that these rules imply that all structure symbols used just for
record structures are placed by default in {\tt usermod}.  This is
usually what a user wants.  But there are times a user might want a
record structure to be associated with the current module.  This can
be used to provide a measure of information hiding: Since no other
module (or usermod) will construct a term associated with this module,
another module can't use unification to look at the subfields inside
such a term.  So in this way, a module can return to a caller a
complex term, and the caller can pass it around and back to the module
in a later call, and only the module code can manipulate that data
structure.\footnote{The hiding is only partial, since other code can
  use {\tt functor/3} or {\tt univ/2} to look inside such terms.  Also
  the very low-level builtin {\tt term\_new\_mod/3} can be used to
  explicitly coerce a term into an arbitrary module.}  The programmer
can tell the compiler to place a particular structure symbol in the
current module by using the {\tt local} directive:

\demo{:- local $Sym/Arity$.}

\noindent which will force all uses of the indicated structure symbol
to be associated with the current module.

An XSB programmer can also export a structure symbol (that is not used
as a predicate), and others can import and use it as a structure
symbol.

Standard predicates (those defined in the XSB environment) are
actually defined in system modules and the compiler implicitly
provides the necessary imports to allow the programmer to use them.
Standard predicates are described in Section \ref{sec:standard}.

For clarity, we state a few consequences of these rules.
\begin{itemize}
\item The module for a particular symbol appearing in a module must be
  uniquely determined.  As a consequence, a symbol of a specific
  $functor/arity$ {\em cannot} be declared as both exported and local,
  or both exported and imported from another module, or declared to be
  imported from more than one module, etc.  These types of environment
  conflicts are detected at compile-time and abort the compilation.
%
\item In \version{}, a module {\em cannot} export a predicate symbol
  that is directly imported from another module, since this would
  require that symbol to be in two modules.  But one can import
  $symbol_1$ from a module {\tt as} $symbol_2$ and then export
  $symbol_2$.  (And $symbol_1$ and $symbol_2$ are allowed to be the
  same symbol.)
%
% Perhaps discuss symbols, if that isn't too confusing.
\index{Prolog flags!{\tt unknown}}
\item If a module {\tt m1} imports a predicate {\tt p/n} from a module
  {\tt m2}, but {\tt m2} does not export {\tt p/n}, nothing is
  detected at the time of compilation.  If {\tt p/n} is defined in
  {\tt m2}, a runtime warning about an environment conflict will be
  issued.  However, if {\tt p/n} is not defined in {\tt m2}, a runtime
  existence error will be thrown~\footnote{This behavior can be
    altered through the Prolog flag {\tt unknown}.}.
\end{itemize}

\subsection{Atoms and 0-Ary Structure Symbols}

XSB uses different internal representations for {\bf atoms} and for
{\bf 0-ary structure symbols}.  Atoms cannot have definitions
associated with them (i.e., cannot be predicates) and are not
associated with modules.  But 0-ary predicates can and are.  The
system automatically coerces atoms to 0-ary structure symbols and vice
versa as necessary.  But when coercing an atom to a 0-ary structure
symbol, it {\bf always} associates the generated structure symbol with
{\tt usermod}.  This can sometimes lead to unexpected results.  As
long as the programmer explicitly exports and imports atoms (or 0-ary
predicate symbols), all works as expected.  But passing an atom as an
argument, and then calling it will always call it in {\tt usermod}.

\subsection{Dynamic Loading and How XSB Finds Code Files} 

When {\tt export} and {\tt import} directives are used, XSB
dynamically (compiles if necessary and) loads the code on demand.
When an imported predicate is called, if the code of the module has
not been loaded into memory, the system finds the code file, compiles
it if necessary, and loads the {\tt .xwam} file into memory.  Then it
invokes the imported predicate.  See Section~\ref{LibPath} for the
details of how the system finds and processes the appropriate XSB
source files.

\subsection{Consulting a Module}
Normally all access to predicates defined in a module is by means of
import declarations.  However, to debug a module it is often
convenient just to consult it at the top-level and then call the
exported predicates with test parameters, which is how non-modules are
handled.  However, note that the predicate to be called after a module
is loaded is in that loaded module, and {\bf not} in {\tt usermod}.  To
allow the programmer to call a predicate exported from the consulted
module without having to explicitly provide the module name, when a
module is consulted, all exported predicates are also defined in {\tt
  usermod} with their same definitions.  (In effect, for exported {\tt
  p/2}, XSB implements {\tt :- import p/2 from $module$ as p/2.} in
{\tt usermod} to provide direct access to {\tt p/2}'s code in $module$
from the {\tt p/2} predicate in {\tt usermod}.)

It is bad form to use this property and consult a module in an
executing program to get access to its exported predicates through
{\tt usermod}.  One should always explicitly import the predicates one
wants to use and let the dynamic loader automatically load the module
code on demand.

\subsection{Usage Inference and the Module System}
The import and export statements of a module $M$ are used by the
compiler for inferring usage of predicates.  At compilation time, if a
predicate $P/N$ occurs as callable in the body of a clause defined in
$M$, but $P$ is neither defined in $M$ nor imported into $M$ from some
other module, a warning is issued that $P/N$ is undefined.  Here
``occurs as callable'' means that $P/N$ is found as a literal in the
body of a clause, or within a system meta-predicate, such as {\tt
assert/1}, {\tt findall/3}, etc.  Currently, occurrences of a term
inside user-defined meta-predicates are not considered as callable by
XSB's usage inference algorithm.  Alternatively, if $P/N$ is defined in
$M$, it is {\em used} if $P/N$ is exported by $M$, or if $P/N$ occurs
as callable in a clause for a predicate that is used in $M$.  The
compiler issues warnings about all unused predicates in a module.  On
the other hand, since all modules are compiled separately, the usage
inference algorithm has no way of checking whether a predicate
imported from a given module is actually exported by that module.

\index{xsbdoc}
\index{declarations!\texttt{document\_export/1}}
\index{declarations!\texttt{document\_import/1}}
Usage inference can be highly useful during code development for
ensuring that all predicates are defined within a set of files, for
eliminating dead code, etc.  In addition, import and export
declarations are used by the {\tt xsbdoc} documentation system to
generate manuals for code.\footnote{Further information on {\tt
xsbdoc} can be found in {\tt \$XSB\_DIR/packages/xsbdoc}.}  For these
reasons, it is sometimes the case that usage inference is desired even
in situations where a given file is not ready to be made into a
module, or it is not appropriate for the file to be a module for some
other reason.  In such a case the directives {\tt document\_export/1}
and {\tt document\_import/1} can be used, and have the same syntax as 
{\tt export/1} and {\tt import/1}, respectively.  These directives
affect only usage inference and {\tt xsbdoc}.  A file is treated as a
module if and only if it includes an {\tt export/1} statement, and
only {\tt import/1} statements affect dynamic loading and name
resolution for predicates.

\subsection{Using Import to Load Usermod Source Files}

When the module system is used to import predicates, code files for
modules are automatically found and dynamically (compiled and) loaded
on first access.  But normally non-module source files must be
explicitly consulted or ensure\_load-ed by some executing program.  To
provide the convenience of dynamic loading to usermod source files,
XSB supports a directive of the form:

\demo{:- import $Pred/Arity$ from usermod($File$).}

\noindent Here $File$ must be the name of a file that contains XSB
source code and is {\bf not} a module, i.e., defines its predicates in
{\tt usermod}.  It must define the predicate $Pred/Arity$.  In this
case, when a goal to $Pred/Arity$ is called and does not yet have a
definition, the file $File$ is (compiled and) loaded, and the goal is
called.  If the predicate already has a definition, that one is
used.\footnote{If the existing definition can be determined to have
  come from a different file, a warning is generated.}

So this facility allows code in non-module files to be treated
somewhat like module files.  But it is the user's responsibility to
ensure that different imported non-module files do not define the same
predicate.  This facility, when carefully used, can eliminate the need
for runtime {\tt consult/1} and {\tt ensure\_loaded/1} commands.  The
form {\tt usermod($File$)} is called a pseudo-module reference, and
can be used in place of module references in import statements.

XSB currently treats:

\demo{:- document\_import $Pred/Arity$ from $File$.}

\noindent as

\demo{:- import $Pred/Arity$ from usermod($File$).}
  
\noindent and a warning is issued if $File$ is {\tt usermod}, since
that should never be a file name.

\subsection{Parameterized Modules in XSB}

The XSB module system now supports parameterized modules: A module can
be parameterized by other modules.  A parameterized module is declared
by including a directive of the form:

\demo{:- module\_parameters($atom_1$, \ldots, $atom_n$).}

\noindent in the module code file.  The atoms, $atom_1$, \ldots,
$atom_n$, are formal module parameters; when the module is loaded,
those module names will be replaced by actual module names passed to
the load operation.  Therefore, module names are now specified by
ground terms: the main structure symbol specifies the base name of the
file containing the module code (as before); the fields of the module
term indicate the names of (the other modules that are) the actual
parameters to the (parameterized) module defined in this file.  Note
that the parameters to modules {\em must} be other modules, and cannot
be constants or any XSB term.  Parameterized modules are a
conservative extension of the former unparameterized module system.

Parameterized modules support a form of higher-order programming which
makes it possible to program some tasks more declaratively.  As a
simple example, consider a module that takes a graph, an initial node
in the graph, and a set of final nodes in the graph, and returns all
final nodes reachable through the graph from the initial node.  A
parameterized module for this task, named {\tt search}, is:
\begin{verbatim}
%% file: search.P

:- module_parameter(example_mod).
:- export reachable_final/1.
:- import initial/1, move/2, final/1 from example_mod.

reachable_final(F) :- reachable(F), final(F).

:- table reachable/1.
reachable(N) :- initial(N).
reachable(N) :- reachable(P), move(P,N).
\end{verbatim}
This search module is parameterized by another module that defines and
exports (at least) 3 predicates: {\tt initial/1}, {\tt move/2}, and
{\tt final/1}.  When this module is loaded, a particular such module
exporting those predicates, must be provided to the loader, and the
formal parameter {\tt example\_mod} will be replaced by that module and
the predicates imported from that module will be used here in the
definitions of {\tt reachable\_final/1} and {\tt reachable/1}.  So
assuming a (non-parameterized) module named {\tt simple\_ex} exports
those 3 predicates, both:
\begin{verbatim}
| ?- import reachable_final/1 from search(simple_ex).
| ?- reachable_final(ReachableFinalState).
\end{verbatim}
and:
\begin{verbatim}
| ?- search(simple_ex):reachable_final(ReachableFinalState).
\end{verbatim}
will return the reachable final states for the problem defined by
{\tt simple\_ex}.

This is second-order in the sense that a module parameter represents a
set of predicates.  Note that this example is (in some sense) fully
declarative, in that there is no explicit procedural code necessary to
load the code for a particular example.  All loading is handled by
XSB's existing dynamic loader.  And this same search module can be
run with many different examples.

Parameterized modules are implemented in XSB as follows.  When a
parameterized module is to be loaded into memory, the formal
parameters are replaced by the actual parameters and that code is
loaded.  (This is actually done by renaming symbols as they are
loaded, so there is minimal effect on loading time.)  This
implementation has two consequences: 1) the performance of code in
parameterized modules is {\em exactly} the same as if the user had
explicitly written the module with the actual parameter modules, and
2) every instance of a parameterized module has its own version of the
module code.  So loading a thousand different instances of a
parameterized module will take a thousand times the space of a single
instance.  In most uses (that I know of so far) this is not a
significant problem, but it should be kept in mind.

One could load another instance of the above module to test the search
algorithm with a different example by:
\begin{verbatim}
| ?- search(hard_ex):reachable_final(ReachableFinalState).
\end{verbatim}
This would load another instance of the search module, named
search(hard\_ex).  Both would be in memory and usable by the user and
by other programs and modules.

So modules in XSB's runtime system are now identified by module names
that are terms, not simply constants.  So anywhere a module name is
required, a parameterized module name, i.e., a module term, can be
used.  The module name must be ground at the time it is required for
use to load specific code, and all structure symbols and atoms in the
structured module name must identify actual files that contain the
appropriate module's code, and those files must be able to be found by
the XSB loader.

To write well-structured and maintainable code, I strongly recommend
that all uses of parameterized modules be done through {\tt import}
directives explicitly appearing in XSB code, and the explicit form of
using the ':' operator to give a module name at runtime be avoided.
(The sole exception is when the user types in such a goal on the
top-level command line.) Using only explicit import directives allows
compile-time analysis of module dependencies which can be critical in
maintaining large XSB code bases. This also requires that the contents
of the {\tt library\_directory/1} predicate can be known at compile
time, which implies programmer discipline in changing that predicate
as well.\footnote{As may be obvious, this has been learned through
  much painful experience. -dsw}.

While parameterized modules can be used in many ways, I think the most
important is in the construction of so-call ``view systems.''  A view
(in the traditional relational database sense) is a relational
operator that takes a set of input relations and views, and produces
an output relation.  By composing views one can build large and
complex systems of data transformations in a completely declarative
way.  With such systems, one often receives base (i.e., input) data
from a source, and then wants to apply a view system to that data,
generating the derived views for use in other applications.  One can
do this declaratively by using parameterized modules.  Each module is
a view definition, exporting the view it defines, and importing the
base and view relations it depends on.  These input relations can be
defined in base modules, and a view module is parameterized by the
base modules it depends on.  Then the same view module can be applied
to the particular input tables obtained from a particular source.

[-dsw: Provide a simple example of a view system...]



